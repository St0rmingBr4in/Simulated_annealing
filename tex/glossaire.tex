\section{Glossaire}

\vspace{4mm}
\makebox[\textwidth][c]{
  \begin{tabular}{|c|L|}

    \hline
    Mot & Définition \\
    \hline
    EDID & Structure de données se situant dans la mémoire EPROM d'un écran qui
    spécifie une résolution que l'écran est capable d'afficher.\\
    \hline
    EPROM & EPROM (Erasable Programmable Read Only Memory) est un type de
    mémoire flash non volatile et réinscriptible.\\
    \hline
    STM32 & Microcontrôleur programmable possédant des entrées/sorties
    permettant de dialoguer via différents protocoles de communication.\\
    \hline
    eMMC & eMMC (embedded Multi-Media Controller) est une puce électronique
    pouvant stocker de l'information.\\
    \hline
    GPU & GPU (Graphical Power Unit) est le processeur graphique.\\
    \hline
    API & API (Application Programming Interface) est un ensemble normalisé de
    classes, de méthodes ou de fonctions qui sert de façade par laquelle un
    logiciel offre des services à d'autres logiciels.\\
    \hline
    bootloader & Un bootloader est le premier programme exécuté par le
    processeur qui s'occupe de lancer le système d'exploitation.\\
    \hline
    CNC & Une machine CNC (Computer numerical control) est une machine à outils
    contrôlés par ordinateur. Cela permet de créer des objets avec une grande
    précision. \\
    \hline
    MIDI & MIDI (Musical Instrument Digital Interface) est un protocole de
    communication utilisé, entre autres, pour la communication entre instruments
    électroniques.\\
    \hline
    SOC & Un SOC (System On Chip), est un système complet embarque sur une seule puce. Par exemple, le
RK3399 est un SOC car il incorpore un processeur, de l’accélération graphique, de la mémoire ainsi
que différentes puces permettant par exemple l’encodage vidéo.\\
    \hline
%    interface série & Une interface série aussi appelée RS232 est un port de
%    communication permettant à deux système d'échanger des données.\\
%    \hline
%    Machine virtuelle & Une machine virtuelle est un PC qui est simulé par un
%    système hôte.\\
%    \hline


  \end{tabular}}

\newpage

