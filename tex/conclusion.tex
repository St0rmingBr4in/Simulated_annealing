\section{Comparaison entre recuit simulé et recherche tabou}
On remarque que notre implémentation de l'algorithme de recherche tabou donne de meilleurs résultats. En effet, on arrive à une meilleure solution en moins de temps. Peut-être que de meilleurs paramètres aurait permis au recuit simulé de donner de meilleurs résultats. L'algorithme de recherche tabou présente l'avantage de nécessiter moins de paramètres à déterminer afin d'obtenir un résultat satisfaisant.
\newpage

\section{Conclusion générale}

Afin de résoudre le problème de placement-routage qui est un problème d'optimisation difficile, nous avons dû utiliser des métaheuristiques. En effet, ce genre de problème
ne peut être résolu facilement et possède souvent de nombreuses solutions de qualité variables. \\

Nous avons implémenté les métaheuristiques recuit simulé et recherche tabou dans le langage C++ afin de résoudre ce problème et de comparer les résultats.\\

Nous avons remarqué que la recherche tabou trouve plus souvent un résultat correct, de plus ce même algorithme converge plus vite.

Ces algorithmes permettent de résoudre plusieurs types de problèmes complexes avec uniquement 
de légères modifications. C'est pourquoi maitriser ces algorithmes et également en connaitre l'existence est important afin de résoudre des problèmes difficiles comme le problème du placement-routage.

\newpage

