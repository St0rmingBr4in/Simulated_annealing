\section{Introduction}

Le problème que nous tentons de résoudre à l'aide de métaheuristique est le problème de placement-routage. \\

Il s'agit d'un problème d'optimisation difficile (NP-hard), où l'objectif est de placer des composants électroniques inter connectés entre eux en réduisant le plus possible les nœuds entre les composants afin de simplifier le routage. Meilleur sera la solution, mieux le circuit électronique fonctionnera. \\

En effet, il s'agit d'un problème NP-hard lorsque les fils en diagonal sont autorisés et NP-complet lorsque uniquement des fils horizontaux ou verticaux ne sont autorisés. Cela implique que non seulement le problème n'est pas toujours soluble mais en plus il peut y avoir plusieurs solutions. \\

De plus les temps de calcul peuvent être extrêmement élevés dans une application réelle et dépendent du nombre de composant et du nombre d'interconnexions. Le nombre de composants est en général très élevé dans des applications réelles. \\

Dans notre cas, les blocs sont disposés sur une grille avec un espacement régulier. Il y a 36 cases et 36 composants.\\

On commence par générer une grille où chaque composant est relié aux composants qui l'entoure puis nous réalisons 1000 changements aléatoires afin de créer le problème que l'on tentera de résoudre à l'aide de divers métaheuristiques.

\newpage

