\section{Présentation de l'entreprise}

\subsection{Le secteur d'activité}

\B regroupe plusieurs secteurs d'activité car l'entreprise a opté pour une
intégration verticale (toutes les étapes techniques critiques sont gérées en
interne au lieu d'être externalisées) ce qui inclus les secteurs de
l'informatique et de l'édition logicielle (apps, site internet, système
d'exploitation) ainsi que la fabrication de matériel électronique (boîtiers,
équipements).

%%% http://pastebin.com/k8C8UG8L


\subsection{L'entreprise}

\B est une start-up française créée en 2015 dont les effectifs ont doublé sur
les six derniers mois et qui a réalisé il y a quelques mois la plus grosse
levée de fonds de l'année en France sur les nouvelles technologies, avec la
somme de 51M\euro. Parmi ses investisseurs clés figurent notamment Pierre
Kosciusko-Morizet (PriceMinister) et Mickael BENABOU (vente-privée).

\B a initialement été créée par 5 associés passionnés de jeux vidéo, qui ont
créé seuls un prototype de streaming d'écran très rapide avec un Linux Debian
modifié fonctionnant sur un PC portable d'entrée de gamme d'il y a 4 ans. Ce
premier pas a permis de lever 3M\euro auprès d'investisseurs privés et a été
rapidement suivi 1 an plus tard par un second tour de table de 10M\euro afin
d'accélérer la croissance, de permettre le financement de l'infrastructure
serveur et le recrutement d'experts.

En juillet, cette année, \B a cette fois ci levé 51M\euro supplémentaires dans le
but de se développer à l'international, plus vite que ses concurrents étrangers
et afin de consolider son avance technologique via des acquisitions
stratégiques et la création de plusieurs pôles d'innovation.

\B compte déjà 5 000 abonnés et 10 000 sur liste d'attente qui seront livrés
dans les prochains mois, aujourd'hui uniquement sur le marché français.
Les offres de \B vont être déployées dans d'autres pays européens très prochainement,
d'abord dans les pays francophones (Luxembourg, Belgique, Suisse) puis ensuite
au Royaume-Uni et en Allemagne.
L'étape suivante est évidemment les États-Unis, où \B a ouvert des bureaux en
Juillet dernier à Palo Alto (au c\oe ur de la Silicon Valley).

%...(les perspective court terme avec le déploiement de la 5G, le plan fibre
%2022 du gouvernement, l'augmentation de la bande passante, de la vitesse, les
%meilleurs algorithmes de compression, H265, VP10, etc)

Demain, \B et ses équipes prévoient de s'ouvrir encre plus à
l'international, ainsi que la création de nouveaux produits et nouvelles
innovations qui permettront d'augmenter les performances et les économies
d'échelle sur toutes ses infrastructures ainsi que d'atteindre les promesses
qu'offrent cette nouvelle technologie : Un monde libéré des contraintes
matérielles, où n'importe quel écran connecté à
Internet dispose d'une puissance illimitée.

\B table sur des prévisions de vente d'au moins 100 000 clients pour 2018
sur tous ces marchés, avec l'ouverture de nouveaux canaux d'acquisition à
travers le marché des professionnels ainsi que l'expansion à de nouveaux pays
étrangers.



\newpage


\subsection{Le service}

Mon stage s'est déroulé au sein de l'équipe \rd chez \s. Cette équipe constitue
un des laboratoires de recherche de \s. Il a été récemment mis en place et
permet le développement de nouvelles solutions hardware pour \s.\\

En effet, \B a mis en place deux ``BLAB'' (``Blade Laboratories''). Ce sont des
laboratoires de recherche visant la création de nouveaux produits, concepts et
technologies (dans le cadre du BLAB hardware) ainsi que de nouveaux algorithmes
(dans le cadre du BLAB software).

Les équipes des BLABs sont dédiées en interne, autonomes, avec pour mission de
délivrer des prototypes fonctionnels tous les 3 à 6 mois.


%scrum
%daily

Le BLAB de \rd est constitué de William, mon maître de stage, Maxime, ainsi que
de Yohann un autre ING2 en stage.

Le BLAB de \rd à eu l'occasion de travailler conjointement avec le service
benchmark afin de tester la latence des services proposés par \B.

\newpage

\subsection{Le positionnement du stage dans l’entreprise}

Le stage a consisté en deux phases principales, la première davantage
``pratique'' a été axée sur l'apprentissage et l'utilisation de la CAO et du
prototypage rapide avec des supports physiques
%(ie. technologies d'impression 3D, découpe laser, découpe jet d'eau, etc \ldots)
et maitrise du logiciel de CAO
Autodesk Fusion360)

Missions :
\begin{itemize}
  \item Modélisation et impression de pièces fonctionnelles
  \item Supports de présentation (pour le salon Paris Games Week)
\end{itemize}

La seconde partie était axée sur le développement logiciel et la fabrication
d'un démonstrateur technologique pour prouver qu'on peut créer un écran de PC
ultra fin qui à la même puissance qu'un PC gamer 8 fois plus lourds et
encombrant

Missions :
\begin{itemize}
  \item Adaptation du client Linux Yocto pour d'autres distributions
  \item Intégration dans différentes boards, RK3328, RK3399
\end{itemize}

Le stage et mes missions étaient intégrées dans le calendrier
événementiel de l'entreprise (\PGW, nouveaux locaux, besoins ponctuels des
différents services, présentation officielle du BLAB)

\newpage

